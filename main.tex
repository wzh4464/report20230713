\documentclass[aspectratio=169]{beamer}
\usepackage{graphicx}
% title
\title{Report on 2023-07-28 Fri}
\author{WU Zihan}

\begin{document}
\maketitle

% First page: report the CUR I understand
\begin{frame}
    \frametitle{CUR}
    \begin{itemize}
        \item $ C = A(:, q) $
        \item $ R = A(p, :) $
        \item $ U = C^{\dagger} A R^{\dagger} $ to minimize $ \| A - CUR \|_F $
    \end{itemize}

    The selection of $ q $ and $ p $ is important. [2] uses the discrete empirical interpolation method (DEIM) and incremental QR decomposition to find the optimal $ q $ and $ p $. And since the n-gram result is sparse, the CUR decomposition can be done in a sparse way as [1] does.

    % a samll bib here from file 'Exported Items.bib'
    % small font
    \scriptsize
    \bibliographystyle{plain}
    \bibliography{ExportedItems}
\end{frame}

% Second page: Last paper
\begin{frame}
    \frametitle{Last Paper}
    I have two valuable talks with Dr. Sheheryar Khan. He suggests me to do several revisements on my last paper.
    \begin{itemize}
        \item Make the diagram more attractive by adding some real pictures in the gerenal framework.
        \item Add some ellipse picture to compatibility matrices figure.
        \item Develop a submethod to add more comparations between different methods.
        \item Some videos can be added to the dataset to be used.
        \item Add more pictures in the metric part and ASSD metric can be added.
    \end{itemize}
    He gave me two papers of his and Ming Zhang's and to behave as examples of how to make diagrams attractive.
    % \scriptsize
    % \bibliographystyle{plain}
    % \bibliography{ExportedItems}
\end{frame}

\end{frame}


\end{document}